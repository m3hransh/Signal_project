\section{Signal Energy}
In this section, we will define a general 
function for calculating the signal energy. 
the total energy in a discrete-time signal 
$x[n]$ over the time interval $n_1 \leq n \leq n_2$ 
is defined as:
\begin{align}
\sum_{n=n_1}^{n_2} |x[n]|^2
\end{align}
\subsection{Q4.1}
The implementation in python is as simple 
as this one line code:

\begin{python}
dt_energy = lambda x, n: sum([np.abs(x(i))**2 for i in n])
\end{python}

\paragraph{}Now let's amend the function to calculate 
the continuous-time signal energy. The total energy 
over the time interval $t_1 \leq t \leq t_2$ in a 
continuous-time signal $x(t)$ is defined as

\begin{align}
\int_{t_1}^{t_2} |x(t)|^2dt
\end{align}
\paragraph{}But in practice, we can't precisely 
calculate the integral, So instead, we can use Riemann sum as follow
\begin{align}
\mathop {\lim }\limits_{n \to \infty } \sum\limits_{i = 1}^n {|x\left( {t_i} \right)|^2\Delta t}
\end{align}
that $n$ is the number of samples and $\Delta t$ is the distance btween samples.

\paragraph{}So for implementation, the only change that 
we should do in comparison to the discrete-time signals 
is to multiply each sample to the distance between them. 
The code is like this:

\begin{python}
# d is the distance between samples
ct_energy = lambda x, n, d: sum([np.abs(x(i))**2 for i in n])*s
\end{python}

\subsection{Q4.2}
 

\pythonexternal[caption={Calculating energy of signals $x(t),y_1(t)$ and $ y_2(t) $}, label={list:q4_2}]{codes/Q4_2.py}

\lstinputlisting[language=bash,style=mystyle,caption={Output of \textbf{Listings \ref{list:q4_2}}}]{codes/Q4_2_out.txt}
